Plan\\
\\
\textbf{Old Structure } \\ 
\\
1. Physical setting \\ 
2. Quasi measurements \\ 
3. Data association step \\
4. Bayesian SLAM \\
5. Numerical results \\
6. Disucssion \\
\\
What was lacking \\
- No point introducing Born's rule. confusing notation \\ 
- No need for equation 16. \\
- (1)and (2)  can be combined into "Physical Setting and Measurements" \\
- No space for likelihood function\\ 
- Map update equation should be front and center \\
\\
\textbf{New Structure} \\
\\
1. \textbf{Physical setting } \\
- describe the state vector (x = (s,r,f)) \\
- sigmoid approximation \\
\\
2. \textbf{Map update function} \\
- turns data into update for f, if other state variables are given (s, r) \\
\\
3. \textbf{Quasi-measurements} \\
- inerleaved between physical measurements \\
\\
4. \textbf{qSLAM} \\
- substitute map update function into a slam bayesan update equation from appendix  \\
\\
5. \textbf{Particle filtering \\}
- Introduce likelihood functions \\
- importance sampling \\ 
- adaptive resampling \\
\\
6. \textbf{ Numerical Results \\}
7. \textbf{ Discussion \\}
8. \textbf{ Conclusion \\}
\\
\\
\textbf{Appendix: SLAM Derivation \\}
- option A: following PF
- option B: start at 30
- finalise notation cos it looks super confusing right now
\\
\\
\textbf{Appendix: Likelihood functions} \\
- for real measurements \\
- for physical measyrements \\
\\