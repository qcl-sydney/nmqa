\section{Recursive Bayesian SLAM Update Equation} \label{sec:appendix1}

	Let $x_t$ be the extended state vector, $z_t$ be sensor data (both physical and quasi-measurements), and let $u_t$ be `controls' such that the choice of the next location for a sensing measurement is specified by an external source. Let $\mathcal{D}_t$ be the set of controls and datasets over the entire procedure, such that $\mathcal{D}_t := \{z_1, u_1, \hdots, z_t \}$. Then the Bayesian posterior can be expanded as:
	\begin{align}
	\prob{}{x_t | \mathcal{D}_t} :&= \prob{}{x_t | z_t, u_{t-1},\mathcal{D}_{t-1}} \\
	&= \frac{\prob{}{z_t | x_t, u_{t-1}, \mathcal{D}_{t-1}} \prob{}{x_t| u_{t-1}, \mathcal{D}_{t-1}} }{\prob{}{ z_t| u_{t-1}, \mathcal{D}_{t-1}}} \label{eqn:bayesfilter:1} \\
	&= \eta_t\prob{}{z_t | x_t, u_{t-1}, \mathcal{D}_{t-1}} \prob{}{x_t| u_{t-1}, \mathcal{D}_{t-1}} \label{eqn:bayesfilter:2} \\
	& = \eta_t\prob{}{z_t | x_t, u_{t-1}} \prob{}{x_t| u_{t-1}, \mathcal{D}_{t-1}} \label{eqn:bayesfilter:3} 
	\end{align} The second last equation follows from the application of Bayes Rule; and we let $\eta_t$ be the normalising constant that is unknown theoretically but a simple normalisation procedure can be implemented numerically. In \cref{eqn:bayesfilter:3}, we assert $x_t, z_t$ are Markov random processes for our application. This means  $x_t, z_t$ do not depend on the entire history of the measurements taken elsewhere on the grid over $[1, t-1)$,  but that all information about the system is available at $x_{t-1}, z_{t-1}$. We invoke this assumption to eliminate $\mathcal{D}_{t-1}$ as a conditional on $z_t$.\\
	To get from \cref{eqn:bayesfilter:3} to  \cref{eqn:recusrivebayesfilter:1}, we reverse marginalise over all possible states at $t-1$. From $t-1$ to $t$, measurement data and the control, $u_{t-1}$ is incorporated and \cref{eqn:recusrivebayesfilter:1} implies the recursive Bayesian update given by \cref{eqn:recusrivebayesfilter:2}. 
	\begin{align}
	\prob{}{x_t | \mathcal{D}_t} &= \eta_t\prob{}{z_t | x_t, u_{t-1}} \int \prob{}{x_t| x_{t-1}, u_{t-1}}  \prob{}{ x_{t-1}| u_{t-1}, \mathcal{D}_{t-1}} dx_{t-1} \label{eqn:recusrivebayesfilter:1} \\
	&= \eta_t\prob{}{z_t | x_t, u_{t-1}} \int \prob{}{x_t| x_{t-1}, u_{t-1}}  \prob{}{ x_{t-1}| \mathcal{D}_{t-1}} dx_{t-1} \label{eqn:recusrivebayesfilter:2}
	\end{align} $\prob{}{x_t| x_{t-1}, u_{t-1}} $ represents known dynamical for the state (equivalently, a Markov transition probability distribution for $x$). The dynamical model is often known for classical motion tracking and map exploration applications, however, we will assign a different physical interpretation to this term in subsequent sections. \\
	\\
	The SLAM problem is introduced explicitly by substituting the sensing qubit coordinates, noise map, and length scales for the true state $x_t := \{s_t, r_t, \map{t} \}$ into \cref{eqn:recusrivebayesfilter:2}. Here, we let $\map{t}:= \{ \mval{j}{t}\}_{1:j:N}$, and we let $r_t :=  \{ \rval{j}{t} \}_{1:j:N}$ exist to approximate $\map{}$ with a true error given by $\epsilon_t$, as given by MyTheorem. \cref{eqn:slam:1} represents the substitution of the extended state vector as sensing qubit coordinates, noise map and length scales respectively.
	\begin{align}
	\prob{}{&s_t, \map{t}, r_t | \mathcal{D}_t} \nonumber \\
	& = \eta_t\prob{}{z_t | s_t, r_t, \map{t}, u_{t-1}} \int \int \int \prob{}{s_{t-1}, r_{t-1}, \map{t-1}| \mathcal{D}_{t-1}} \prob{}{ s_t, r_t, \map{t}| s_{t-1}, r_{t-1}, \map{t-1}, u_{t-1}}  ds_{t-1} dr_{t-1} d\map{t-1}  \label{eqn:slam:1} 
	\end{align}
	We can expand the second term in \cref{eqn:slam:1} using the product rule. 
	\begin{align}
	\prob{}{ s_t, r_t, \map{t}| s_{t-1}, r_{t-1}, \map{t-1},u_{t-1}}  = \prob{}{ s_t| r_t, \map{t}, s_{t-1}, r_{t-1}, \map{t-1},u_{t-1}} \prob{}{ r_t, \map{t}| s_{t-1}, r_{t-1}, \map{t-1},u_{t-1}} \label{eqn:slam:2} 
	\end{align}
	\cref{eqn:slam:2} can be simplified to \cref{eqn:slam:3} if the effect of sensing qubits does not depend on the map, or the lengthscales. Similarly, if the noise field evolves independently of the availability of the qubits for sensing, then we can use \cref{eqn:slam:4}.
	\begin{align}
	\prob{}{ s_t, r_t, \map{t}| s_{t-1}, r_{t-1}, \map{t-1},u_{t-1}}  &= \prob{}{ s_t| s_{t-1}, u_{t-1}} \prob{}{ r_t, \map{t}| s_{t-1}, r_{t-1}, \map{t-1},u_{t-1}} \label{eqn:slam:3} \\
	&= \prob{}{ s_t| s_{t-1}, u_{t-1}} \prob{}{ r_t, \map{t}|  r_{t-1}, \map{t-1}} \label{eqn:slam:4}  \\
	&= \prob{}{ s_t| s_{t-1}, u_{t-1}} \prob{}{  \map{t}|  r_t, r_{t-1}, \map{t-1}} \prob{}{ r_t|  r_{t-1}, \map{t-1}} \label{eqn:slam:5} \\
	&= \prob{}{ s_t| s_{t-1}, u_{t-1}} \prob{}{ \map{t}|  \map{t-1}} \prob{}{ r_t|  r_{t-1}, \map{t-1}} \label{eqn:slam:6} 
	\end{align} In \cref{eqn:slam:4}, we have two Markov transition probabilities, or dynamic models:  $\prob{}{ s_t| s_{t-1}, u_{t-1}}$ for the choice of qubits for sensing, and $\prob{}{ r_t, \map{t}|  r_{t-1}, \map{t-1}}$ for time evolution of a noise field and the implied evolution of associated length-scales for the system. In \cref{eqn:slam:5}, we separate the natural evolution of the environment with any dynamical update we design for the choice of length-scales using the product rule. The natural evolution $\prob{}{  \map{t}|  r_t, r_{t-1}, \map{t-1}} $  is independent of lengthscales for the system, yielding a Markov transition probability $\prob{}{ \map{t}|  \map{t-1}}$ in \cref{eqn:slam:6}. The second term, $\prob{}{ r_t|  r_{t-1}, \map{t-1}}$ represents our choice of model for dynamically updating length-scales as the environment evolves. Substitution of  \cref{eqn:slam:6}  into \cref{eqn:slam:1} yields:
	\begin{align}
	&\prob{}{s_t, \map{t}, r_t | \mathcal{D}_t} \nonumber \\
	& = \eta_t\prob{}{z_t | s_t, r_t, \map{t}, u_{t-1}} \int \prob{}{ s_t| s_{t-1}, u_{t-1}}  \int \prob{}{ \map{t}|  \map{t-1}} \int \prob{}{ r_t|  r_{t-1}, \map{t-1}} \prob{}{s_{t-1}, r_{t-1}, \map{t-1}| \mathcal{D}_{t-1}}  ds_{t-1} d\map{t-1} dr_{t-1}   \label{eqn:slam:7} 
	\end{align} 
	In the case that $\map{}$ represents a stationary, perfectly correlated process in the time domain, we can assume that both $\map{t} = \map{} \forall t$, This logically suggests that $r$ used for the approximation of $\map{}$ should also be time invariant, $r_t = r \forall t$. Hence, going from $t-1$ to $t$ means that the integral over $\map{t-1}$, $r_{t-1}$ will be zero everywhere unless $\map{t}=\map{t-1}, r_t = r_{t-1}$. 
	Under these assumptions for $\map{}, r$ in the time domain, we reduce \cref{eqn:slam:7} to:
	\begin{align}
	&\prob{}{s_t, \map{t}, r_t | \mathcal{D}_t} = \eta_t\prob{}{z_t | x_t, u_{t-1}}\quad  \times \nonumber \\
	& \int \prob{}{ s_t| s_{t-1}, u_{t-1}}  \int \prob{}{ \map{t}|  \map{t-1}} \int \delta(\map{t} - \map{t-1})  \delta(r_{t} - r_{t-1})  \prob{}{ r_t|  r_{t-1}, \map{t-1}} \prob{}{s_{t-1}, r_{t-1}, \map{t-1}| \mathcal{D}_{t-1}}  ds_{t-1} d\map{t-1} dr_{t-1}   \label{eqn:slam:8}  \\
	& = \eta_t\prob{}{z_t | x_t, u_{t-1}} \int \prob{}{ s_t| s_{t-1}, u_{t-1}} \int \delta(r_{t} - r_{t-1})  \prob{}{ r_t|  r_{t-1}, \map{t}} \prob{}{s_{t-1}, r_{t-1}, \map{t}| \mathcal{D}_{t-1}}  ds_{t-1}  dr_{t-1}   \label{eqn:slam:9}  \\
	& = \eta_t\prob{}{z_t | x_t, u_{t-1}} \int \prob{}{ s_t| s_{t-1}, u_{t-1}} \prob{}{s_{t-1}, r_t, \map{t}| \mathcal{D}_{t-1}}  ds_{t-1}    \label{eqn:slam:10}
	\end{align} 
	Under perfect time domain correlations for the noise field and time invariant length-scales, \cref{eqn:slam:7} reduces to \cref{eqn:slam:11} and the latter is directly analogous to the central problem outlined in \cite{thrun2001probabilistic}
	