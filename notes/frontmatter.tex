
\title{Efficient scheduling of noise characterization protocols in quantum computing architectures}

\author{Riddhi Swaroop Gupta} 
\email{riddhi.sw@gmail.com}
\affiliation{ARC Centre of Excellence for Engineered Quantum Systems, School of Physics, The University of Sydney, New South Wales 2006, Australia}

\author{Michael J. Biercuk}
\affiliation{ARC Centre of Excellence for Engineered Quantum Systems, School of Physics, The University of Sydney, New South Wales 2006, Australia}

\begin{abstract}
Spectator qubits embedded in quantum computing architectures enable in situ detection of noise processes affecting quantum hardware. Common spatial correlations between processing and spectator qubits present a new resource which may be exploited in efficient scheduling of measurements. We present an algorithmic framework for 2D field mapping in quantum computing architectures using sparse measurements. We adapt classical simultaneous localisation and mapping (SLAM) techniques to enable spatial field characterisation using idle qubits (hence, QSLAM), where idle qubits are a static or dynamically available resource. We implement QSLAM via a particle filter that shares information between neighbouring qubits while discovering neighbourhood sizes relevant to the system; an adaptive controller then schedules future measurements based on this algorithm. We use experimental measurements on a linear array of trapped ions subject to an observed but uncontrolled magnetic field gradient (noise). Numeric simulations demonstrate that QSLAM outperforms a brute force approach for estimating the magnetic field gradient by over an order of magnitude across a range of operating parameter regimes. Extensions include incorporating time dynamics in field evolution and/or sensor availability.
\end{abstract}

\maketitle
